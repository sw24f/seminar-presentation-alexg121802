% Options for packages loaded elsewhere
\PassOptionsToPackage{unicode}{hyperref}
\PassOptionsToPackage{hyphens}{url}
%
\documentclass[
  ignorenonframetext,
  aspectratio=169,
]{beamer}
\usepackage{pgfpages}
\setbeamertemplate{caption}[numbered]
\setbeamertemplate{caption label separator}{: }
\setbeamercolor{caption name}{fg=normal text.fg}
\beamertemplatenavigationsymbolsempty
% Prevent slide breaks in the middle of a paragraph
\widowpenalties 1 10000
\raggedbottom
\setbeamertemplate{part page}{
  \centering
  \begin{beamercolorbox}[sep=16pt,center]{part title}
    \usebeamerfont{part title}\insertpart\par
  \end{beamercolorbox}
}
\setbeamertemplate{section page}{
  \centering
  \begin{beamercolorbox}[sep=12pt,center]{part title}
    \usebeamerfont{section title}\insertsection\par
  \end{beamercolorbox}
}
\setbeamertemplate{subsection page}{
  \centering
  \begin{beamercolorbox}[sep=8pt,center]{part title}
    \usebeamerfont{subsection title}\insertsubsection\par
  \end{beamercolorbox}
}
\AtBeginPart{
  \frame{\partpage}
}
\AtBeginSection{
  \ifbibliography
  \else
    \frame{\sectionpage}
  \fi
}
\AtBeginSubsection{
  \frame{\subsectionpage}
}
\usepackage{amsmath,amssymb}
\usepackage{iftex}
\ifPDFTeX
  \usepackage[T1]{fontenc}
  \usepackage[utf8]{inputenc}
  \usepackage{textcomp} % provide euro and other symbols
\else % if luatex or xetex
  \usepackage{unicode-math} % this also loads fontspec
  \defaultfontfeatures{Scale=MatchLowercase}
  \defaultfontfeatures[\rmfamily]{Ligatures=TeX,Scale=1}
\fi
\usepackage{lmodern}
\ifPDFTeX\else
  % xetex/luatex font selection
\fi
% Use upquote if available, for straight quotes in verbatim environments
\IfFileExists{upquote.sty}{\usepackage{upquote}}{}
\IfFileExists{microtype.sty}{% use microtype if available
  \usepackage[]{microtype}
  \UseMicrotypeSet[protrusion]{basicmath} % disable protrusion for tt fonts
}{}
\makeatletter
\@ifundefined{KOMAClassName}{% if non-KOMA class
  \IfFileExists{parskip.sty}{%
    \usepackage{parskip}
  }{% else
    \setlength{\parindent}{0pt}
    \setlength{\parskip}{6pt plus 2pt minus 1pt}}
}{% if KOMA class
  \KOMAoptions{parskip=half}}
\makeatother
\usepackage{xcolor}
\newif\ifbibliography
\setlength{\emergencystretch}{3em} % prevent overfull lines
\providecommand{\tightlist}{%
  \setlength{\itemsep}{0pt}\setlength{\parskip}{0pt}}
\setcounter{secnumdepth}{-\maxdimen} % remove section numbering
\usepackage{natbib}
\usepackage{amsmath}
\usepackage{graphicx}
\usepackage{booktabs}
\usepackage{setspace}
\usepackage{float}
\usepackage{hyperref}
\setbeamersize{text margin left=1in, text margin right=1in}
\ifLuaTeX
  \usepackage{selnolig}  % disable illegal ligatures
\fi
\IfFileExists{bookmark.sty}{\usepackage{bookmark}}{\usepackage{hyperref}}
\IfFileExists{xurl.sty}{\usepackage{xurl}}{} % add URL line breaks if available
\urlstyle{same}
\hypersetup{
  pdftitle={Effect sizes for paired data should use the change score variability rather than the pre-test variability.},
  pdfauthor={Alex Gould},
  hidelinks,
  pdfcreator={LaTeX via pandoc}}

\title{Effect sizes for paired data should use the change score
variability rather than the pre-test variability.}
\author{Alex Gould}
\date{2024-10-02}

\begin{document}
\frame{\titlepage}

\begin{frame}{Abstract}
\protect\hypertarget{abstract}{}
\begin{block}{Full Citation}
\protect\hypertarget{full-citation}{}
Dankel, SJ and Loenneke, JP. Effect sizes for paired data should use the
change score variability rather than the pre-test variability. J
Strength Cond Res 35(6): 1773--1778, 2021---
\end{block}

\begin{block}{What is effect size}
\protect\hypertarget{what-is-effect-size}{}
\begin{itemize}
\tightlist
\item
  Variable that provides an overall measure for magnitude of change
  (Cite)
\item
  Differs from a T-statistic because sample size is not included
\item
  Used in various baseline-post-treatment comparison

  \begin{itemize}
  \tightlist
  \item
    Specifically, they are looking at this comparison from the lens of
    meta-analyses for exercise science and sports medicine
  \end{itemize}
\end{itemize}
\end{block}
\end{frame}

\begin{frame}{Main Issue}
\protect\hypertarget{main-issue}{}
In effect size calculations for paired data,the authors argue that
people tend to use the wrong kind of variability. What are the two kinds
of variability that they are referring to?
\end{frame}

\begin{frame}{Variability of the Study Sample}
\protect\hypertarget{variability-of-the-study-sample}{}
\begin{itemize}
\tightlist
\item
  Any measure of difference between subjects in a given treatment group
\item
  Represented by the Baseline and Post-treatment Standard Deviation.
\item
  Dankel(Cite) and his team claim that the use of this type of
  variability in paired-sample studies is useless as it has nothing to
  do with the treatment itself
\end{itemize}
\end{frame}

\begin{frame}{Variability of the Intervention}
\protect\hypertarget{variability-of-the-intervention}{}
\begin{itemize}
\tightlist
\item
  Any measure of difference between baseline and post-treatment measure
\item
  Represented in this case by the Standard Deviation of Change Scores (I
  will elaborate on this later)
\item
  Dankel and his team prefer this method of assessing variability
\end{itemize}
\end{frame}

\begin{frame}{Authors}
\protect\hypertarget{authors}{}
\begin{block}{Dr.~Scott Dankel}
\protect\hypertarget{dr.-scott-dankel}{}
\begin{itemize}
\tightlist
\item
  Professor at Rowan University, a public research university in New
  Jersey
\item
  Attended the University of Mississippi to pursue a Masters and PhD in
  Exercise Science
\item
  Research Interests include acute and chronic adaptations to blood flow
  restricted exercise (Cite)
\end{itemize}
\end{block}
\end{frame}

\begin{frame}{Authors (continued\ldots)}
\protect\hypertarget{authors-continued}{}
\begin{block}{Jeremy Paul Loenneke}
\protect\hypertarget{jeremy-paul-loenneke}{}
\begin{itemize}
\tightlist
\item
  Professor at The University of Mississippi
\item
  Attended Southeast Missouri State for his Bachelors and Masters in
  Nutrition and Exercise Science
\item
  Eventually got his PhD in Exercise Physiology at the University of
  Oklahoma
\item
  Research Discipline is in Skeletal Muscle Plasticity (Cite)
\end{itemize}
\end{block}

\begin{block}{General Comments}
\protect\hypertarget{general-comments}{}
\begin{itemize}
\tightlist
\item
  Regarding the disciplines of the authors, this paper was published in
  The Journal of Strength and Conditioning Research
\item
  Good example of the use of statistics as an interdisciplinary tool
\end{itemize}
\end{block}
\end{frame}

\begin{frame}{Introduction}
\protect\hypertarget{introduction}{}
The author's claim that the common effect size measures listed below are
used exhaustively in meta-analyses in the exercise science discipline.

\begin{block}{Specific Effect Size Measures}
\protect\hypertarget{specific-effect-size-measures}{}
\begin{itemize}
\item
  Cohen's d (Cite)
\item
  Hedge's g (Cite)
\item
  Glass delta (Cite)
\item
  Each use some combination of baseline standard deviation and
  post-treatment standard deviation.
\item
  Measures of variability of the study sample
\end{itemize}
\end{block}
\end{frame}

\begin{frame}{Paired Data vs.~Independent Data}
\protect\hypertarget{paired-data-vs.-independent-data}{}
\begin{block}{Independent Data}
\protect\hypertarget{independent-data}{}
\begin{itemize}
\tightlist
\item
  Data collected through an Independent design

  \begin{itemize}
  \tightlist
  \item
    Each subject is only measured once
  \item
    Subjects are allocated into a baseline group and a post-treatment
    group
  \item
    Study sample variability is more important
  \item
    The pooled standard error is the way to assess this variability
  \end{itemize}
\end{itemize}
\end{block}

\begin{block}{Paired Data}
\protect\hypertarget{paired-data}{}
\begin{itemize}
\tightlist
\item
  Data that is collected through a Paired Sample design

  \begin{itemize}
  \tightlist
  \item
    The same subject is assessed at both the baseline and post-treatment
    time points.
  \item
    Since its based on the same subject, this data is not independent
  \item
    In this type of Design, study sample variability is irrelevant
  \item
    Variability of the Intervention assessed by standard error of the
    change scores
  \end{itemize}
\end{itemize}
\end{block}
\end{frame}

\begin{frame}{Methods}
\protect\hypertarget{methods}{}
\end{frame}

\begin{frame}{Figure 1}
\protect\hypertarget{figure-1}{}
\end{frame}

\begin{frame}{Figure 2}
\protect\hypertarget{figure-2}{}
\end{frame}

\begin{frame}{Results}
\protect\hypertarget{results}{}
\end{frame}

\begin{frame}{Discussion}
\protect\hypertarget{discussion}{}
\end{frame}

\begin{frame}{References}
\protect\hypertarget{references}{}
\end{frame}

\end{document}
