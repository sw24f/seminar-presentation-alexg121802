% Options for packages loaded elsewhere
\PassOptionsToPackage{unicode}{hyperref}
\PassOptionsToPackage{hyphens}{url}
%
\documentclass[
]{article}
\usepackage{amsmath,amssymb}
\usepackage{iftex}
\ifPDFTeX
  \usepackage[T1]{fontenc}
  \usepackage[utf8]{inputenc}
  \usepackage{textcomp} % provide euro and other symbols
\else % if luatex or xetex
  \usepackage{unicode-math} % this also loads fontspec
  \defaultfontfeatures{Scale=MatchLowercase}
  \defaultfontfeatures[\rmfamily]{Ligatures=TeX,Scale=1}
\fi
\usepackage{lmodern}
\ifPDFTeX\else
  % xetex/luatex font selection
\fi
% Use upquote if available, for straight quotes in verbatim environments
\IfFileExists{upquote.sty}{\usepackage{upquote}}{}
\IfFileExists{microtype.sty}{% use microtype if available
  \usepackage[]{microtype}
  \UseMicrotypeSet[protrusion]{basicmath} % disable protrusion for tt fonts
}{}
\makeatletter
\@ifundefined{KOMAClassName}{% if non-KOMA class
  \IfFileExists{parskip.sty}{%
    \usepackage{parskip}
  }{% else
    \setlength{\parindent}{0pt}
    \setlength{\parskip}{6pt plus 2pt minus 1pt}}
}{% if KOMA class
  \KOMAoptions{parskip=half}}
\makeatother
\usepackage{xcolor}
\usepackage[margin=1in]{geometry}
\usepackage{graphicx}
\makeatletter
\def\maxwidth{\ifdim\Gin@nat@width>\linewidth\linewidth\else\Gin@nat@width\fi}
\def\maxheight{\ifdim\Gin@nat@height>\textheight\textheight\else\Gin@nat@height\fi}
\makeatother
% Scale images if necessary, so that they will not overflow the page
% margins by default, and it is still possible to overwrite the defaults
% using explicit options in \includegraphics[width, height, ...]{}
\setkeys{Gin}{width=\maxwidth,height=\maxheight,keepaspectratio}
% Set default figure placement to htbp
\makeatletter
\def\fps@figure{htbp}
\makeatother
\setlength{\emergencystretch}{3em} % prevent overfull lines
\providecommand{\tightlist}{%
  \setlength{\itemsep}{0pt}\setlength{\parskip}{0pt}}
\setcounter{secnumdepth}{-\maxdimen} % remove section numbering
\ifLuaTeX
  \usepackage{selnolig}  % disable illegal ligatures
\fi
\IfFileExists{bookmark.sty}{\usepackage{bookmark}}{\usepackage{hyperref}}
\IfFileExists{xurl.sty}{\usepackage{xurl}}{} % add URL line breaks if available
\urlstyle{same}
\hypersetup{
  pdftitle={Abstract for October 7th Presentation},
  pdfauthor={Alex Gould},
  hidelinks,
  pdfcreator={LaTeX via pandoc}}

\title{Abstract for October 7th Presentation}
\author{Alex Gould}
\date{2024-09-29}

\begin{document}
\maketitle

\doublespacing

I decided to present the methods paper entitled: ``Effect Sizes for
Paired Data Should Use the Change Score Variability Rather Than the
Pre-test Variability'' By Scott J. Dankel and Jeremy P. Loenneke of the
University of Mississippi. This paper had two main objectives: to
convince the audience that pre-test and post-test standard deviations
don't always tell the full story on the overall variability of the
dataset, as well as to convince the audience that the heterogeneity of
the study sample can play a part in influencing effect size measurements
and in turn paired t-test results.

Through both objectives, Dankel and Loenneke advocated for the use of
the effect size measurement Cohen's \(d_z\) over measurements like
Hedge's \(g\), Cohen's \(d\), and Glass's \(\delta\) which use the likes
of pre and post-test standard deviation to formulate their estimate. I
believe that they made a well-informed argument for the most part
regarding these topics, and I am excited to lay it out in presentation
form, although, some of the measurements that they used did not line up
with the data that is reported in their figures.

Specifically, this was an issue with the standard deviation of the
change scores (which was the measurement that built the foundation for
this paper). Dankel and his team ironically define the measure using the
correct equation later in the paper after using the wrong metric for the
analysis. I plan to re-analyze the standard deviation of the change
scores using the aforementioned equation and through my own sources to
provide more accurate results, and to see if the significance that they
reported still holds.

I found this paper interesting because it deals with the idea of
baseline and change scores in a different way than how I plan on
exploring similar topics in my thesis. Overall, I am excited to present
the complexities of their argument as well as the strengths and
shortcomings of their analysis on Monday, October 7th at 5:30 pm in AUST
247.

Keywords: Cohen's \(d\), Hedge's \(g\), Paired \(t\) test, paired-sample
testing, change-score mean, standard deviation of the change scores

\end{document}
